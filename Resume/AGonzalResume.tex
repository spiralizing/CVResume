%%%%%%%%%%%%%%%%%
% This is an sample CV template created using altacv.cls
% (v1.6.4, 13 Nov 2021) written by LianTze Lim (liantze@gmail.com). Now compiles with pdfLaTeX, XeLaTeX and LuaLaTeX.
%
%% It may be distributed and/or modified under the
%% conditions of the LaTeX Project Public License, either version 1.3
%% of this license or (at your option) any later version.
%% The latest version of this license is in
%%    http://www.latex-project.org/lppl.txt
%% and version 1.3 or later is part of all distributions of LaTeX
%% version 2003/12/01 or later.
%%%%%%%%%%%%%%%%

%% Use the "normalphoto" option if you want a normal photo instead of cropped to a circle
% \documentclass[10pt,a4paper,normalphoto]{altacv}

\documentclass[10pt,a4paper,ragged2e,withhyper]{altacv}
%% AltaCV uses the fontawesome5 and packages.
%% See http://texdoc.net/pkg/fontawesome5 for full list of symbols.

% Change the page layout if you need to
\geometry{left=1.25cm,right=1.25cm,top=1.5cm,bottom=1.5cm,columnsep=1.2cm}

% The paracol package lets you typeset columns of text in parallel
\usepackage{paracol}

% Change the font if you want to, depending on whether
% you're using pdflatex or xelatex/lualatex
\ifxetexorluatex
  % If using xelatex or lualatex:
  \setmainfont{Roboto Slab}
  \setsansfont{Lato}
  \renewcommand{\familydefault}{\sfdefault}
\else
  % If using pdflatex:
  \usepackage[rm]{roboto}
  \usepackage[defaultsans]{lato}
  % \usepackage{sourcesanspro}
  \renewcommand{\familydefault}{\sfdefault}
\fi

% Change the colours if you want to
\definecolor{SlateGrey}{HTML}{2E2E2E}
\definecolor{LightGrey}{HTML}{666666}
\definecolor{DarkPastelRed}{HTML}{06326F}
\definecolor{PastelRed}{HTML}{3A95A9}
\definecolor{GoldenEarth}{HTML}{E7D192}
\colorlet{name}{black}
\colorlet{tagline}{PastelRed}
\colorlet{heading}{DarkPastelRed}
\colorlet{headingrule}{GoldenEarth}
\colorlet{subheading}{PastelRed}
\colorlet{accent}{PastelRed}
\colorlet{emphasis}{SlateGrey}
\colorlet{body}{LightGrey}

% Change some fonts, if necessary
\renewcommand{\namefont}{\Huge\rmfamily\bfseries}
\renewcommand{\personalinfofont}{\footnotesize}
\renewcommand{\cvsectionfont}{\LARGE\rmfamily\bfseries}
\renewcommand{\cvsubsectionfont}{\large\bfseries}


% Change the bullets for itemize and rating marker
% for \cvskill if you want to
\renewcommand{\itemmarker}{{\small\textbullet}}
\renewcommand{\ratingmarker}{\faCircle}

%% Use (and optionally edit if necessary) this .tex if you
%% want to use an author-year reference style like APA(6)
%% for your publication list
% When using APA6 if you need more author names to be listed
% because you're e.g. the 12th author, add apamaxprtauth=12
\usepackage[backend=biber,style=apa6,sorting=ydnt]{biblatex}
\defbibheading{pubtype}{\cvsubsection{#1}}
\renewcommand{\bibsetup}{\vspace*{-\baselineskip}}
\AtEveryBibitem{\makebox[\bibhang][l]{\itemmarker}}
\setlength{\bibitemsep}{0.25\baselineskip}
\setlength{\bibhang}{1.25em}


%% Use (and optionally edit if necessary) this .tex if you
%% want an originally numerical reference style like IEEE
%% for your publication list
% \usepackage[backend=biber,style=ieee,sorting=ydnt]{biblatex}
%% For removing numbering entirely when using a numeric style
\setlength{\bibhang}{1.25em}
\DeclareFieldFormat{labelnumberwidth}{\makebox[\bibhang][l]{\itemmarker}}
\setlength{\biblabelsep}{0pt}
\defbibheading{pubtype}{\cvsubsection{#1}}
\renewcommand{\bibsetup}{\vspace*{-\baselineskip}}


%% sample.bib contains your publications
\addbibresource{sample.bib}

\begin{document}
\name{Alfredo Gonz\'alez-Espinoza}
\tagline{PhD / Postdoctoral Researcher}
%% You can add multiple photos on the left or right
\photoR{2.8cm}{Mypic}
% \photoL{2.5cm}{Yacht_High,Suitcase_High}

\personalinfo{%
  % Not all of these are required!
  \email{spiralizing@gmail.com}
  \phone{484-758-8636}
  \mailaddress{4814 Chester Ave.}
  \location{Philadelphia, PA, USA}
  \homepage{agonzaleze.wordpress.com}
  \twitter{@spiralizing}
  \linkedin{spiralizing}
  \github{spiralizing}
  \orcid{0000-0003-2361-1827}
  %% You can add your own arbitrary detail with
  %% \printinfo{symbol}{detail}[optional hyperlink prefix]
  % \printinfo{\faPaw}{Hey ho!}[https://example.com/]
  %% Or you can declare your own field with
  %% \NewInfoFiled{fieldname}{symbol}[optional hyperlink prefix] and use it:
  % \NewInfoField{gitlab}{\faGitlab}[https://gitlab.com/]
  % \gitlab{your_id}
  %%
  %% For services and platforms like Mastodon where there isn't a
  %% straightforward relation between the user ID/nickname and the hyperlink,
  %% you can use \printinfo directly e.g.
  % \printinfo{\faMastodon}{@username@instace}[https://instance.url/@username]
  %% But if you absolutely want to create new dedicated info fields for
  %% such platforms, then use \NewInfoField* with a star:
  % \NewInfoField*{mastodon}{\faMastodon}
  %% then you can use \mastodon, with TWO arguments where the 2nd argument is
  %% the full hyperlink.
  % \mastodon{@username@instance}{https://instance.url/@username}
}

\makecvheader
%% Depending on your tastes, you may want to make fonts of itemize environments slightly smaller
% \AtBeginEnvironment{itemize}{\small}

%% Set the left/right column width ratio to 6:4.
\columnratio{0.6}

% Start a 2-column paracol. Both the left and right columns will automatically
% break across pages if things get too long.
\begin{paracol}{2}
\cvsection{Experience}

\cvevent{Postdoctoral Researcher}{University of Pennsylvania}{April 2019 -- Present}{Philadelphia, USA}
\begin{itemize}
\item I am a postdoctoral researcher in the mathematical biology group
\item I study different aspects of music, from structural properties to innovation and cultural evolution
\end{itemize}

\divider

\cvevent{Postdoctoral Researcher}{National Institute of Genomic Medicine}{October 2018 -- March 2019}{Mexico City, Mexico}
\begin{itemize}
\item I worked in the computational genomics lab
\item I studied gene-expression correlations in breast cancer
\end{itemize}

\cvsection{Projects I have worked in}

\cvevent{Evolution and innovation in music scores}{University of Pennsylvania}{}{}
I process and analyze thousands of music scores in digital format, looking for trends in innovation and characterizing composers and musical periods using tools from stochastic processes and information theory.


\divider

\cvevent{Cultural evolution in music listening histories}{University of Pennsylvania}{}{}

I constructed time series of listening histories (songs of what people listen) from LastFM platform consisting in more than 600k users and more than 3Billion of listenings. We use the data to quantify frequency dependent selection with an inference methodology to understand how people choose what they listen.

\divider

\cvevent{Gene co-expression in breast cancer}{National Institute of Genomic Medicine}{}{}
I developed a hybrid clustering method, using tools from random matrix theory and the $k-$medoids algorithm. We used gene expression data for breast cancer to identify groups of genes that behave similar, finding patterns in different chromosomes and malignancy of the disease.

\divider

\cvevent{Nonlinearity and irreversibility in music scores}{Institute of Physical Sciences}{}{}

During my PhD I analyzed thousands of music scores, constructing univariate and multivariate time series. Using tools and concepts from statistical physics and time series analysis we characterize the music scores identifying patterns over the years.

\medskip

\cvsection{Personal Interests}

% Adapted from @Jake's answer from http://tex.stackexchange.com/a/82729/226
% \wheelchart{outer radius}{inner radius}{
% comma-separated list of value/text width/color/detail}
\wheelchart{1.5cm}{0.5cm}{%
  6/8em/accent!30/{Music, Movies \& Art},
  3/8em/accent!40/Amateur cellist,
  8/8em/accent!60/Fun questions to do research in,
  2/10em/accent/Video games,
  5/6em/accent!20/Outdoor \\ activities \\ (mostly soccer)
}

% use ONLY \newpage if you want to force a page break for
% ONLY the current column
%\newpage

\cvsection{Publications}

\nocite{*}
%\bibliography{sample.bib}
\printbibliography[heading=pubtype,title={\printinfo{\faFile*[regular]}{Peer-reviewed papers}},type=article]

\divider

%% Switch to the right column. This will now automatically move to the second
%% page if the content is too long.
\switchcolumn

\cvsection{My Life Philosophy}

\begin{quote}
``Enjoy every moment in life, have fun in what you do and never lose curiosity.''
\end{quote}

\cvsection{Strengths/Skills}

\cvtag{Creative thinker}
\cvtag{Eager/Fast learner}\\
\cvtag{Driven by curiosity}

\divider\smallskip

\cvtag{Julia}
\cvtag{Computer Simulations}
\cvtag{Linux}
\\
\cvtag{Python}
\cvtag{Statistical Analysis}
\cvtag{Inference}
\\
\cvtag{NLP}
\cvtag{Time Series Analysis}
\cvtag{Bash}
\\
\cvtag{Hypothesis Testing}
\cvtag{Machine Learning}

\divider\smallskip

\cvevent{Programming skills/packages:}{}{}{}

Bash, SSH, Git, SQLite, PostgreSQL, JSON, Spark, Jupyter notebook, Pluto notebook. {\bf Python:} Pandas, Numpy, PyPlot, Keras, TensorFlow, Pytorch, Scikit-learn, NLTK, scipy, music21. {\bf Julia:} DataFrames, Flux, Plots. 

\cvsection{Languages}

\cvskill{Spanish}{5}
\divider

\cvskill{English}{4.5}
\divider


%% Yeah I didn't spend too much time making all the
%% spacing consistent... sorry. Use \smallskip, \medskip,
%% \bigskip, \vspace etc to make adjustments.
\medskip

\cvsection{Education}

\cvevent{Ph.D.\ in Science (Physics)}{Universidad Autónoma del Estado de Morelos (UAEM)}{September 2014 -- August 2018}{}
Thesis title: Music scores characterization from a complex systems perspective

\divider

\cvevent{M.Sc.\ in Science (Physics)}{UAEM}{Sept 2011 -- June 2014}{}
Thesis title: A discrete model for the Liesegang-type pattern formation in the reaction {\em NH$_3 +$HCl}
\divider

\cvevent{B.Sc.\ in Science (Chemistry)}{UAEM}{Sept 2005 -- June 2011}{}
Thesis title: An {\em ab-initio} molecular potential for hydroxylamine

\divider

\medskip
% \newpage
\cvsection{References}

% \cvref{name}{email}{mailing address}
\cvref{Prof.\ Joshua B. Plotkin}{University of Pennsylvania}{jplotkin@sas.upenn.edu}
%{Address Line 1\\Address line 2}

\divider

\cvref{Prof.\ Gustavo Mart\'inez-Mekler}{Institute of Physical Sciences, UNAM}{mekler@icf.unam.mx}
%{Address Line 1\\Address line 2}


\end{paracol}


\end{document}
